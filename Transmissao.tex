\documentclass[a4paper,12pt]{article}
\usepackage[utf8]{inputenc}
\usepackage[brazil]{babel}
\usepackage[margin = 1in]{geometry}
\usepackage{amsmath}
\usepackage{graphicx}
\usepackage{subfigure}
\usepackage{listings}
\usepackage{amssymb}
\usepackage{array}
\usepackage{indentfirst}
\usepackage{textcomp}




\begin{document}
  
\begin{titlepage}
 \vfill
  \begin{center}
   {\large \textbf{UNIVERSIDADE FEDERAL DO PARANÁ}} \\[5cm]

   {\large {Marco Antonio Rios  GRR20133243 \\ Thierry Dompsin Maurice GRR20133243 }}\\[5cm]


   {\Large \textbf{TRANSMISSÃO DE UM JOGO NA ARENA DA BAIXADA}}\\[4cm]
  \vfill

\vspace{2cm}

\large \textbf{Curitiba}

\large \textbf{Junho de 2015}
\end{center}
\end{titlepage}
  
  
  
  
  \thispagestyle{empty}
  \clearpage
  \setcounter{page}{1}
  \tableofcontents
  
  \clearpage
  
  \section{INTRODUÇÃO}
  
  Para que um jogo de futebol seja transmitido para as casas dos torcedores,
ocorre uma grande mobilização das emissoras de TV para que ocorra a melhor
transmissão possível. Cerca de 10 a 20 câmeras são usadas (variando de jogo a
jogo), sendo estas fixas, ou móveis; mais de cem pessoas são mobilizadas, entre
elas, reportes, cinegrafistas, narradores, entre outros.

As câmeras e os equipamentos eletrônicos são conectados a um caminhão
da emissora estacionado fora do estádio por fios de até 300 metros. Nesta unidade
móvel, trabalham 20 pessoas que selecionam e preparam os replays que são
captados pelas câmeras e microfones dentro do campo.

Apenas uma imagem por vez é enviada à emissora pela antena do caminhão.
No prédio desta, uma antena capta o sinal do satélite; as micro-ondas são
decodificadas em sinal de TV e enviado para uma sala onde a imagem será pós-produzida.

Ainda na emissora, há a sala do switcher onde ocorre a “mixagem” das
imagens que irão para a casa dos espectadores. No intervalo do jogo, por exemplo,
alterna-se a imagem do estádio com outras produzidas nos estúdios ou videotapes.
Na sala do switcher são inseridos, ainda, os GCs (as paralavras que aparecem na
tela informando o placar, as substituições, escalações etc.). Tudo que será realizado
antes, durante e depois do jogo é pré-programado, sendo uma pessoa responsável
pela atualizações das estatísticas e outra nas publicações na internet em tempo
real. Um coordenador opera as atividades dentro do switcher usando ponto
eletrônico para facilitar a comunicação com a equipe de cerca de 20 pessoas.

Do switcher, a imagem é codificada em sinal de VHF é transmitida pela
mesma antena que recebe os sinais via satélite. Mesmo com toda essa operação
para se transmitir um jogo, um gol feito no estádio chega às casas dos fans com
atraso de 1 segundo.
 
  \subsection{APRESENTAÇÃO DO PROBLEMA}
A emissão de um jogo de Futebol entre Atlético-PR e Vasco da Gama do Rio
de Janeiro, na Baixada, deve ser gerado por um equipamento rádio móvel e, então,
primeiramente deve ser enviado para uma torre de TV que deverá distribuir este
sinal na região metropolitana de Curitiba a uma distância máxima de 50km nas
frequências de 3.5 GHz, 8.45GHz e 12GHz. Este mesmo sinal, a partir da torre de
TV, em sua frequência máxima (12GHz), deve ser enviado para um satélite
geoestacionário que dista 43.00km e então em canal aberto ser entregue para uma
torre de televisão na Gávea que deverá distribuí-lo para todo o estado do Rio de
Janeiro.

Fornece-se para a solução deste projeto:
\begin{itemize}
 \item  potência de transmissão do rádio móvel = $32W$;
 \item  potência de transmissão do satélite para Terra =$ 0.35kW$;
 \item  tensão induzida de Geração = $12.4 dB \mu V$;
 \item  nível mínimo de sinal aceitável = $-36dBm$;
 \item  nível mínimo do sinal na Gávea = $-79dBm$;
 \item  característica das antenas: as antenas utilizadas são antenas direcionais, cujo ganho não é unitário ($0dB$);
 \item  diâmetro da antena para o rádio móvel $D=1.47m$;
 \item  diâmetro da antena para a torre $d=1.3m$;
 \item  eficiência da antena = 72%;
 \item  área efetiva da antena $Ae=0.4m2$;
\end{itemize}
Toda a propagação se dá em espaço livre.
 \subsection{OBJETIVO}
Apresentar detalhadamente uma solução para a transmissão de um jogo
entre Atlético e Vasco.
 \subsubsection{OBJETIVO GERAL}
Dado o princípio da transmissão de um sinal e os seus devidos problemas na
distribuição, captação e envio desta, objetivou-se desenvolver uma possível solução
para o Projeto.

 \subsubsection{OBJETIVOS ESPECÍFICOS}
Dentre os principais objetivos destacam-se:
\begin{itemize}
  \item cálculo do ganho das antenas;
  \item cálculo as perdas por propagação;
  \item cálculo a perda da transmissão entre as antenas;
  \item cálculo da potência recebida (Pr) na torre de TV;
  \item cálculo da margem de sinal recebido (M);
  \item análise do sinal Terra-satélite;
  \item cálculo da atenuação em espaço livre;
  \item cálculo da potência recebida no satélite;
  \item cálculo da densidade de fluxo do sinal em espaço-livre;
  \item cálculo do e.i.r.p.;
  \item cálculo da densidade de fluxo Poyinting na recepção;
  \item cálculo da margem do sinal recebido (M) na Gávea;
\end{itemize}
 
 \section{ANÁLISE DO SINAL EM TERRA}
 \subsection{CÁLCULO DO GANHO DAS ANTENAS}
  Sabe-se que a potência recebida pela antena isotrópica:
  \begin{equation}
  \dfrac{P_t}{4 \pi d^2} \times \dfrac{\lambda ^ 2}{4 \pi}
  \end{equation}

e a potência recebida pela antena parabólica

\begin{equation}
  \dfrac{P_t}{4 \pi d^2} \times Ae
\end{equation}

 onde
 
 \begin{equation}
  Ae = \dfrac{\pi D^2}{4} \times \eta
 \end{equation}
 
 representa a abertura efetiva da antena. Logo, tem-se que o ganho G representa a
relação entre a potência recebida pela antena parabólica com a potência recebida
pela antena isotrópica. Assim,

\begin{equation}
 G = \dfrac{\dfrac{P_t}{4 \pi d^2} \times Ae}{ \dfrac{P_t}{4 \pi d^2} \times \dfrac{\lambda ^ 2}{4 \pi}} = \dfrac{4 \pi Ae}{\lambda^2} 
\end{equation}

consequentemente o ganho em dBi será

\begin{equation}
 G[dBi] = 10 \times log(\dfrac{4 \pi Ae}{\lambda^2})
\end{equation}

Então, para o rádio móvel, sendo D=1.47m e $\eta = 0.72$, tem-se

\begin{equation}
 Ae = \dfrac{\pi D^2}{4} \times = \dfrac{\pi \times 1.47^2}{4} \times 0.72 = 1.221960162 m^2
\end{equation}

Utilizando-se o a equação 05, substitui-se o valor de Ae da equação 06 e os três
valores de frequência (3.5GHz, 8.45GHz e 12GHz), obtém-se os valores da tabela
1.
\begin{table}[!h]
  \begin{center}
    \caption{Valores calculados dos ganhos para diferentes frequências} 
    \begin{tabular}{p{3in} p{3in}}\hline \hline
      \multicolumn{1}{c}{frequência(GHz)} & \multicolumn{1}{c}{Ganho G(dBi)}\\ \hline
      \multicolumn{1}{c}{3.5} & \multicolumn{1}{c}{33.19} \\ 
      \multicolumn{1}{c}{8.45} & \multicolumn{1}{c}{40.85} \\ 
      \multicolumn{1}{c}{12} & \multicolumn{1}{c}{43.90}\\ \hline
      &
    \end{tabular}
  \end{center}
\end{table}


Agora, para a Torre, sendo d=1.3m e $\eta = 0.72$, tem-se

\begin{equation}
 Ae = \dfrac{\pi D^2}{4} \times \eta = \dfrac{\pi \times 1.3^2}{4} \times 0.72 = 0.95 m^2
\end{equation}

Utilizando-se o a equação (5), substitui-se o valor de Ae da equação (7) e os três
valores de frequência (3.5GHz, 8.45GHz e 12GHz), obtém-se os valores da tabela
2.

\begin{table}[!h]
  \begin{center}
    \caption{Valores calculados dos ganhos para diferentes frequências} 
    \begin{tabular}{p{3in} p{3in}}\hline \hline
      \multicolumn{1}{c}{frequência(GHz)} & \multicolumn{1}{c}{Ganho G(dBi)}\\ \hline
      \multicolumn{1}{c}{3.5} & \multicolumn{1}{c}{32.10} \\ 
      \multicolumn{1}{c}{8.45} & \multicolumn{1}{c}{40.16} \\ 
      \multicolumn{1}{c}{12} & \multicolumn{1}{c}{42.81}\\ \hline
      &
    \end{tabular}
  \end{center}
\end{table}

Desta maneira, o ganho total será

\begin{equation}
 G_{TOTAL} = G_{RADIOMOVEL} + G_{TORRE}
\end{equation}

Logo, para cada valor de frequência obtém-se, a partir da equação 08, diferentes
ganhos como mostra a tabela 3.

\begin{table}[!h]
  \begin{center}
    \caption{Valores calculados dos ganhos para diferentes frequências} 
    \begin{tabular}{p{1in} p{3.5in} p{3in}}\hline \hline
      \multicolumn{1}{c}{frequência(GHz)} & \multicolumn{1}{c}{Soma dos Ganhos (Rádio Móvel e Torre) (dBi)} & \multicolumn{1}{c}{Ganho Total(dBi)}\\ \hline
      \multicolumn{1}{c}{3.5} & \multicolumn{1}{c}{33.19+32.10} & \multicolumn{1}{c}{65.29} \\ 
      \multicolumn{1}{c}{8.45} & \multicolumn{1}{c}{40.85+40.16} & \multicolumn{1}{c}{81.36} \\ 
      \multicolumn{1}{c}{12} & \multicolumn{1}{c}{43.90+42.81} & \multicolumn{1}{c}{86.71}\\ \hline
      &
    \end{tabular}
  \end{center}
\end{table}

\subsection{CÁLCULO DAS PERDAS DE PROPAGAÇÃO}
Para se obter as perdas por propagação, dado que d=50km, utiliza-se a
seguinte fórmula

\begin{equation}
 T_o = 92.44 + 20 \times log(f)(Ghz) + 20 \times log(d)(km)
 \end{equation}

 Substituiu-se os valores de frequência (3.5GHz, 8.45 GHz e 12GHz) na equação 09,
têm-se os valores das perdas de propagação na tabela 4.

\begin{table}[!h]
  \begin{center}
    \caption{Valores calculados das perdas por propagação} 
    \begin{tabular}{p{3in} p{3in}}\hline \hline 
      \multicolumn{1}{c}{frequência(GHz)} & \multicolumn{1}{c}{Perdas $T_0$(dBi}\\ \hline
      \multicolumn{1}{c}{3.5} & \multicolumn{1}{c}{137.3} \\ 
      \multicolumn{1}{c}{8.45} & \multicolumn{1}{c}{144.3} \\ 
      \multicolumn{1}{c}{12} & \multicolumn{1}{c}{148}\\ \hline
      &
    \end{tabular}
  \end{center}
\end{table}

\subsection{CÁLCULO DAS PERDAS DE TRANSMISSÃO ENTRE ANTENAS}

Sendo as perdas de transmissão entre as antenas igual a diferença entre o
as perdas por Propagação e o ganho da Torre, tem-se

\begin{equation}
\Xi_o = T_o - G_{TORRE} 
\end{equation}
Com os valores obtidos para as perdas de transmissão da tabela 4 e os ganhos da
Torre da tabela 2, substituiu-se estes na equação 10. Assim, obtemos os valores
das perdas de transmissão entre as antenas na tabela 5.

\begin{table}[!h]
  \begin{center}
    \caption{Valores calculados das perdas de transmissão entre antenas} 
    \begin{tabular}{p{1in} p{1in} p{1in} p{1in}}\hline \hline
      \multicolumn{1}{c}{frequência} & \multicolumn{1}{c}{Ganho da Torre} & \multicolumn{1}{c}{Perdas por Propagação} & \multicolumn{1}{c}{Perdas por transmissão} \\ 
      \multicolumn{1}{c}{(Ghz)} & \multicolumn{1}{c}{(dB)} & \multicolumn{1}{c}{(dB)} & \multicolumn{1}{c}{(dB)} \\ \hline
      \multicolumn{1}{c}{3.5} & \multicolumn{1}{c}{65.29} & \multicolumn{1}{c}{137.3} & \multicolumn{1}{c}{72.01} \\ 
      \multicolumn{1}{c}{8.45} & \multicolumn{1}{c}{81.36} & \multicolumn{1}{c}{144.3} & \multicolumn{1}{c}{62.94} \\ 
      \multicolumn{1}{c}{12} & \multicolumn{1}{c}{86.71} & \multicolumn{1}{c}{148} & \multicolumn{1}{c}{61.29} \\ \hline
      &
    \end{tabular}
  \end{center}
\end{table}

\subsection{CÁLCULO DA POTÊNCIA RECEBIDA NA TORRE DE TV}
O valor da potência recebida pela torre de TV é dada pela equação

\begin{equation}
 P_R = P_o + \Xi _o
\end{equation}

Dado que a potência do transmissor do rádio móvel é de 32W, têm-se que esse
valor em dBm 

\begin{equation}
P[dBm]= 10 \times log\dfrac{P_T}{1mW} = 10 \times log\frac{32}{1mW} = 45dBm 
\end{equation}

Desta maneira, pode-se calcular o valor da potência recebida pela torre de Tv. Os
valores obtidos encontram-se na tabela 6.

\begin{table}[!h]
  \begin{center}
    \caption{Valores calculados das potências recebidas na torre de TV} 
    \begin{tabular}{p{3in} p{3in}}\hline \hline
      \multicolumn{1}{c}{frequência(GHz)} & \multicolumn{1}{c}{Perdas $P_R$ (dBm)}\\ \hline
      \multicolumn{1}{c}{3.5} & \multicolumn{1}{c}{-27.01} \\ 
      \multicolumn{1}{c}{8.45} & \multicolumn{1}{c}{-17.94} \\ 
      \multicolumn{1}{c}{12} & \multicolumn{1}{c}{-16.29}\\ \hline
      &
    \end{tabular}
  \end{center}
\end{table}

Percebe-se que os valores obtidos estão abaixo do nível mínimo de sinal a ser
recebido na torre de TV (-36dBm). Desta maneira, deve-se calcular a margem de
sinal recebido (M).

\subsection{CÁLCULO DA MARGEM DE SINAL RECEBIDO}

Sendo -36dBm o nível mínimo de sinal a ser recebido na torre de TV, podese
calcular a margem de sinal recebido através da equação

\begin{equation}
 M = P_R -(-36dBm)
\end{equation}
  Assim, para diferentes valores de frequência, têm-se os valores obtidos para a
margem de sinal na tabela 7.

\begin{table}[!h]
  \begin{center}
    \caption{Valores calculados da margem de sinal recebido} 
    \begin{tabular}{p{3in} p{3in}}\hline \hline
      \multicolumn{1}{c}{frequência(GHz)} & \multicolumn{1}{c}{Margem M (dBm)}\\ \hline
      \multicolumn{1}{c}{3.5} & \multicolumn{1}{c}{8.99} \\ 
      \multicolumn{1}{c}{8.45} & \multicolumn{1}{c}{18.06} \\ 
      \multicolumn{1}{c}{12} & \multicolumn{1}{c}{19.71}\\ \hline
      &
    \end{tabular}
  \end{center}
\end{table}

Fazendo a comparação da Margem M obtida na tabela 7 com os valores mínimos,
nota-se que esses estão acima como mostra a tabela 8.

\begin{table}[!h]
  \begin{center}
    \caption{Comparação das Margens} 
    \begin{tabular}{p{3in} p{3in}}\hline \hline
      \multicolumn{1}{c}{Margem M} & \multicolumn{1}{c}{Margem M (dBm)}\\ \hline
      \multicolumn{1}{c}{8.99} & \multicolumn{1}{c}{$M \leq 20*log f = 10.88$} \\ 
      \multicolumn{1}{c}{18.06} & \multicolumn{1}{c}{$M \leq 20*log f = 18.53$} \\ 
      \multicolumn{1}{c}{19.71} & \multicolumn{1}{c}{$M \leq 20*log f = 21.58$}\\ \hline
      &
    \end{tabular}
  \end{center}
\end{table}

Para efeito prático, a margem do sinal recebido deve ser $M \leq 20*log$ f para que não
se entregue apenas ruído. Logo todos os sinais entregues são satisfatórios.

O esquemático da figura 1 exemplifica o problema até agora tratado

\section{ANÁLISE TERRA-SATÉLITE}
\subsection{ANÁLSE DO SINAL TERRA-SATÉLITE}
A análise pode-se ser vista segundo a figura 2, onde nota-se a torre de TV
mandando o sinal de 12GHz para o satélite.
\subsection{CÁLCULO DA ATENUAÇÃO EM ESPAÇO LIVRE}
Sendo a equação 09, pode-se obter, também, a atenuação em espaço livre. 
\begin{equation}
 A_{EL} = 92.44 + 20 \times log(f)(GHz) + 20 \times log(d)(km)
\end{equation}
Sendo os parâmetros de f e d iguais a 12GHz e 43000km, respectivamente, tem-se
o valor da atenuação dado por
\begin{equation}
 A_{EL} = 92.44 + 20 \times log(12) + 20 \times log(43000) = 206.69dB
\end{equation}

\subsection{POTÊNCIA RECEBIDA NO SATÉLITE}
A potência é dada pela equação
\begin{equation}
 P_{RSAT} = P_{T-TORRE} - A_{EL} + G_{RX-SATELITE}
 \end{equation}

Deseja-se saber a potência recebida pela Torre $(P_T)$ e a Ganho recebido pelo
satélite $(G_{RX})$, sendo que $A_{EL}$ já foi calculado e tem o valor de 206.69dB

\subsection{CÁLCULO DO GANHO DA ANTENA SATELITAL G$_{RX}$}
Para se calcular o ganho da antena, primeiramente, deve-se obter o
comprimento de onda para uma frequência de 12GHz.

\begin{equation}
 \lambda = \dfrac{c}{f} = \dfrac{3 \times 10^8}{12 \times 10^9} = 0.025
\end{equation}
Com o valor do comprimento de onda obtido na equação 16 e o valor dado da área
do satélite igual a $0.4m^2$, pode-se obter o ganho através da equação 5.

\begin{equation}
 G_{RX} = 10 \times log \left (\dfrac{4 \pi A_e}{\lambda^2} \right ) = 39.05dBi = 7943W
\end{equation}

Considera-se que a potência de transmissão da torre de TV para o satélite é de
55dBm. Logo, tem se a potência recebida pelo satélite dada por
 
 \begin{equation}
  P_{RSAT} = 55dBm - 206.69dB + 39.05dBi = - 112.64dBm
 \end{equation}
Assim, a potência recebida no satélite pode ser calculada utilizando a fórmula de
Friis (1946).

\begin{equation}
 P_{RSAT} = \left (\dfrac{\lambda}{4 \pi d} \right) ^2 \times G_{TX} \times G_{RX} \times P_T
\end{equation}

sendo PT a potência transmitida pela torre de TV.
Desta maneira, tem-se o esquemático da figura 3.

\section{ANÁLISE SATÉLITE-TERRA}
\subsection{CÁLCULO DA DENSIDADE DE FLUXO DO SINAL EM ESPAÇO-LIVRE (FLUXO DE VETOR DE POYNTING)}
O fluxo de vetor de Poynting é calculado pela fórmula
\begin{equation}
 P_{POYNTING} = \dfrac{P_t}{4 \pi d} = \dfrac{350}{4 \pi \times 43000^2} = 1.50 \times 10^{-8} W
 \end{equation}
 
 Logo, o valor em dB do fluxo do vetor de Poynting
 
\begin{equation}
 P_{POYNTING}[dB] = 20 \times log(P_{POYNTING}) = -156.44 \dfrac{dBw}{m^2} 
\end{equation}

\subsection{CÁLCULO DO E.I.R.P}
O cálculo do e.i.r.p. significa potência efetiva irradiada. Deve-se calcular o
e.i.r.p. toda vez que se liga um transmissor a uma antena, para se saber qual a real
potência que a antena está transmitindo. O fato de se querer saber qual a potência
real é devido à se achar parte da potência que se perde nos polos além do restante
da potência sofre a atenuação do ganho da antena.
Logo, pode-se calcular o e.i.r.p. através da equação

\begin{equation}
 e.i.r.p = P_t + G_t - P
\end{equation}
onde as incógnitas são:
\begin{itemize}
  \item  Pt = Potência transmitida = 0.35kW = 25.45dBW;
  \item Gt = Ganho de transmissão = 39.05 dB;
  \item P = Perda nos cabos = 4dB;
\end{itemize}

Nota-se que a e.i.r.p. deve ser calculada em dBm ou em dBW, Assim, 

\begin{equation}
e.i.r.p = 25.45dBW + 39.05dB - 4dB = 60dBW
\end{equation}

O valor da equação em dBm será

\begin{equation}
 e.i.r.p = 65.44dBm + 39.05dB - 4dB = 100.49dBm
\end{equation}

\subsection{DENSIDADE DE FLUXO P$_{POYNTING}$ NA RECEPÇÃO}
O valor da densidade do fluxo de Poynting na recepção é calculado por

\begin{equation}
  P_{POYNTING-RX} = e.i.r.p +  P_{POYNTING}
\end{equation}

substituindo os valores obtidos das equações 22 e 25 para fluxo do vetor de
Poynting e e.i.r.p., respectivamente, tem-se

\begin{equation}
 P_{POYNTING-RX} = -156.44dBw +  60dBW = -96.44\dfrac{dBw}{m^2}
 \end{equation}
 
 ou 
 
 \begin{equation}
  P_{POYNTING-RX} = -69.88 \dfrac{dBmW}{m^2}
 \end{equation}

 \subsection{CÁLCULO DA MARGEM DO SINAL RECEBIDO (M) NA GÁVEA}
 Assim, substituindo o valor obtido na equação 28, a margem do sinal que é
recebido na Gávea é calculado por

\begin{equation}
 M = P_R - (-79dBm) = -69.88 - (-79) = 9.12
\end{equation}
Sendo a margem de sinal calculada por 
\begin{equation}
 M \leq 20 \times log (f) = 20 \times log 12 \therefore M \leq 21.58
\end{equation}

 
\section{CONCLUSÃO} 
O sinal recebido na Gávea é de -69.88 dBmW que é altamente satisfatório,
pois está abaixo do sinal mínimo de -79dBm que foi imposta no projeto. O que
representa 9.12 dB acima do mínimo.
Pode-se verificar pela simulação da figura 4 que, dado os parâmetros de
e.i.r.p., a distância entre as antenas e entre outros, pode-se trilhar, assim, um
caminho que o sinal irá percorrer.
 
 
 
 

\end{document}

